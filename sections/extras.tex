\section{Extras}
\begin{frame}[fragile]
  \slidetitle
  This section covers some advanced topics:
  \begin{itemize}
    \item Cleaning workspace
    \item Working with patches
    \item Search keywords
    \item ...
  \end{itemize}
\end{frame}

\subsection{What to commit and what not to commit}
\begin{frame}[fragile]
  \slidetitle
  It's best practice to not commit:
  \begin{itemize}
    \item Generated Binaries
    \item Large files (Use \lstinline{git-lfs})
    \item Your personal editor configuration
    \item passwords and other secrets
    \item ...
  \end{itemize}

  Note: Make use of the \lstinline{.gitignore} file!

\end{frame}

\subsection{Remove untracked files}
\begin{frame}[fragile]
    \subslidetitle
  Tracking a project under git helps you to keep your workspace clean, after your compilation process generated some temporary files:

  \begin{lstlisting}
(*\textcolor[HTML]{18B2B2}{(master)}*) $ (*\textcolor[HTML]{0000AA}{touch \$(echo "compile" | hexdump | head -n1)}*)
(*\textcolor[HTML]{18B2B2}{(master)}*) $ (*\textcolor[HTML]{0000AA}{git status}*)
On branch master
...
(*\textcolor[HTML]{AA0000}{ 0000000}*)
(*\textcolor[HTML]{AA0000}{ 0a65}*)
(*\textcolor[HTML]{AA0000}{ 6c69}*)
(*\textcolor[HTML]{AA0000}{ 6f63}*)
(*\textcolor[HTML]{AA0000}{ 706d}*)
...
\end{lstlisting}
  Now we would like to clean all this temporary files, using \cmd{git clean}:
  \begin{lstlisting}
(*\textcolor[HTML]{18B2B2}{(master)}*) $ (*\textcolor[HTML]{0000AA}{git clean -df}*)
Removing 0000000
Removing 0a65
...
\end{lstlisting}

\end{frame}

\subsection{Git protocols}
\begin{frame}[fragile]
  \subslidetitle
  Git repositories can be accessed locally or over the network.
  \\
  \vspace{1em}
  Various protocols are supported:
  \begin{itemize}
  \opt{{\bf ssh}} {{\bf normally used for read-write access}}
  \opt{http[s]} {good for read only access without password}
  \opt{local}   {file system based}
  \opt{git}     {git native protocol on port 9418}
  \opt{legacy}  {ftp, rsync, ...}
  \end{itemize}
  \vspace{1em}

  SSH provides us the possibility to authenticate without having to enter the password on every request.
\end{frame}

\subsection{Configure SSH access}
\begin{frame}[fragile]
  \subslidetitle
  Create a SSH key pair:
  \begin{lstlisting}
$ (*\textcolor[HTML]{0000AA}{ssh-keygen -f \textasciitilde/.ssh/id\_rsa}*)
Generating public/private rsa key pair.
Enter passphrase (empty for no passphrase): (*\textcolor[HTML]{0000AA}{<enter>}*)
Enter same passphrase again: (*\textcolor[HTML]{0000AA}{<enter>}*)
Your identification has been saved in .../.ssh/id_rsa.
Your public key has been saved in .../.ssh/id_rsa.pub.
The key fingerprint is:
7e:f8:15:2a:b3:a2:9c:30:4e:c7:60:50:a4:d5:a9:82 user@host
The key's randomart image is:
+--[ RSA 2048]----+
|   . . .         |
|  . = = S        |
|   = X * X O o   |
...
\end{lstlisting}
\end{frame}

\subsection{Configure SSH access}
\begin{frame}[fragile]
  \subslidetitle
  Display your public key:
  \begin{lstlisting}
$ (*\textcolor[HTML]{0000AA}{cat \textasciitilde/.ssh/id\_rsa.pub}*)
ssh-rsa AAAAB3NzaC1yc2EAAAADAQABAAABAQDOmt7Y4H51gc2m
GmZsFzES6shVLFLEJ/lFCTwyosWHYDaluK71nGCelp61oTocgf4N
HBwTZmo0EZ1k0RHYt8Q3LF8e5fbC+dXt5E35XtkVFuUC7IG2/6fm
NW41j3lw9UUVrOBDgx+QvvoCuRQaxNd4mRaLsRbj9WXt17hGuNNW
ioKPWLSpw/4KHJ34hCrnliAQJ+jlW/0ieOooFp057diCka6Jn7BW
jXHi8sWMxIfyPyV2+4Kt8OpChFNYjzaL5LMRRhMnvJ8zP5SFJB2q
HP50zPYQ+gKoSda7GZedZRgD7gT7ir/u8X9HSpNyTNTafhp9+3Aj
uUiYLTgtczTgYk/T user@host
\end{lstlisting}

  This whole output can be added to the SSH access keys
  section in the web frontend of your git appliance.
\end{frame}

\subsection{Search with git}
\begin{frame}[fragile]
  \subslidetitle
  Lost in search? It is just convenient to have \cmd{git grep}:
  \begin{lstlisting}
(*\textcolor[HTML]{18B2B2}{(master)}*) $ (*\textcolor[HTML]{0000AA}{git grep moon.js}*)
moon.html:        <script src="(*\textcolor[HTML]{AA0000}{moon.js}*)"></script>
\end{lstlisting}

  Or we can search commits that touch lines containing the keyword you are looking for using \cmd{git log --pickaxe-regexp}:
  \begin{lstlisting}
(*\textcolor[HTML]{18B2B2}{(master)}*) $ (*\textcolor[HTML]{0000AA}{git log --oneline --abbrev --pickaxe-regex "*moon*"}*)
(*\textcolor[HTML]{ae6617}{a3c399f}*) remove the blue moon
(*\textcolor[HTML]{ae6617}{93ea12c}*) change the green moon color to red
...
39719c9 initial commit
\end{lstlisting}

  Finally a option of \cmd{git branch} to search the branch a commit belongs to:
  \begin{lstlisting}
(*\textcolor[HTML]{18B2B2}{(master)}*) $ (*\textcolor[HTML]{0000AA}{git branch --contains a3c399f}*)
(*\textcolor[HTML]{00AA00}{* master}*)
\end{lstlisting}
\end{frame}

\subsection{Finger pointing!?}
\begin{frame}[fragile]
  \subslidetitle

  The \cmd{git blame} returns revision and author information per each line of a given file:
  \begin{lstlisting}
(*\textcolor[HTML]{18B2B2}{(master)}*) $ (*\textcolor[HTML]{0000AA}{git blame moon.js}*)
...
0cf42e1 (Timo Furrer <DATE>  22) // create moons
0cf42e1 (Timo Furrer <DATE>  23) new Moon("green");
0ba92c4 (Timo Furrer <DATE>  24) new Moon("gray");
0e213fb (Timo Furrer <DATE>  25) new Moon("white");
...
\end{lstlisting}

\end{frame}

\subsection{Work with patches}
\begin{frame}[fragile]
    \subslidetitle
  Remove the blue moon:
  \begin{lstlisting}
(*\textcolor[HTML]{18B2B2}{(master)}*) $ (*\textcolor[HTML]{0000AA}{sed -i "/blue/d" moon.js}*)
(*\textcolor[HTML]{18B2B2}{(master)}*) $ (*\textcolor[HTML]{0000AA}{git commit -a -m "remove the blue moon"}*)
\end{lstlisting}

  The \cmd{git format-patch} generates patch file to be send to 3rd party collaborator, or mailing list:
  \begin{lstlisting}
(*\textcolor[HTML]{18B2B2}{(master)}*) $ (*\textcolor[HTML]{0000AA}{git format-patch -1}*)
0001-remove-the-blue-moon.patch
\end{lstlisting}

  \begin{lstlisting}
(*\textcolor[HTML]{18B2B2}{(master)}*) $ (*\textcolor[HTML]{0000AA}{cat 0001-remove-the-blue-moon.patch}*)
From 7afa035a39c2dc9648b182772eee06e3181ae24e Mon Sep 17 00:00:00 2001
From: Eric Keller <keller.eric@gmail.com>
Date: Sun, 29 Nov 2015 10:06:27 +0000
Subject: [PATCH] remove the blue moon
...
 // create moons
 new Moon("green");
-new Moon("blue");
 new Moon("white");
...
\end{lstlisting}

\end{frame}

\subsection{Apply a patch}
\begin{frame}[fragile]
    \subslidetitle
  First remove the last commit:
  \begin{lstlisting}
(*\textcolor[HTML]{18B2B2}{(master)}*) $ (*\textcolor[HTML]{0000AA}{git reset --hard HEAD\textasciicircum1}*)
\end{lstlisting}
  Create new no-blue branch in order to apply this patch
  \begin{lstlisting}
(*\textcolor[HTML]{18B2B2}{(master)}*) $ (*\textcolor[HTML]{0000AA}{git switch -c no-blue}*)
Switched to a new branch 'no-blue'
\end{lstlisting}

  The \cmd{git am} command takes a list of patch to apply to the current working area:
  \begin{lstlisting}
(*\textcolor[HTML]{18B2B2}{(no-blue)}*) $ (*\textcolor[HTML]{0000AA}{git am -3 0001-remove-the-blue-moon.patch}*)
Applying: remove the blue moon
\end{lstlisting}

  Note: As well as when merging branch the \cmd{git am} could potentially end up with conflict, therefore we would like to enforce a 3way-merge with the \cmd{-3} option.

\end{frame}

\subsection{Advanced undo}
\begin{frame}[fragile]
    \subslidetitle

  Want to undo a rebase operation? You deleted an important file? Got interrupted in the middle of an octopus merge... Do not worry \cmd{git reflog} is here to help.

  \begin{lstlisting}
(*\textcolor[HTML]{18B2B2}{(master)}*) $ (*\textcolor[HTML]{0000AA}{git reflog}*)
(*\textcolor[HTML]{ae6617}{9ed88d3}*) HEAD@{0}: checkout: moving from death-star to master
(*\textcolor[HTML]{ae6617}{429131d}*) HEAD@{1}: rebase finished: returning to refs/heads/death-star
(*\textcolor[HTML]{ae6617}{429131d}*) HEAD@{2}: rebase: add death-star
(*\textcolor[HTML]{ae6617}{9ed88d3}*) HEAD@{3}: rebase: checkout master
(*\textcolor[HTML]{ae6617}{9dd4b4b}*) HEAD@{4}: checkout: moving from master to death-star
(*\textcolor[HTML]{ae6617}{9ed88d3}*) HEAD@{5}: commit: snow white comment
\end{lstlisting}

  You use reflog in combination with the \cmd{git reset --hard HEAD@\{4\}} command.
\end{frame}

\subsection{Git submodules}
\begin{frame}[fragile]
  \subslidetitle
  Git offers the possibility to include other git repositories in a subdirectory of a project.
  \\
  \vspace{1em}
  The following commands deal with submodules:
  \begin{itemize}
    \item \cmd{git clone URL --recursive}
    \item \cmd{git submodule add URL PATH}
    \item \cmd{git submodule update}
    \item \cmd{git submodule sync}
  \end{itemize}
  \vspace{1em}
  Note: the main project can commit the subdirectory and fix the commit of the submodule.
\end{frame}

\subsection{Git aliases}
\begin{frame}[fragile]
  \subslidetitle

  Git allows to create aliases for commands.
  \\
  \vspace{1em}
  Example:
  instead of typing \cmd{git status} we want to type \cmd{git st}.
  \\
  \vspace{1em}
  Define aliases:

  \begin{lstlisting}
  git config --global alias.st = status
  git config --global alias.sw = switch
  git config --global alias.ci = commit
  git config --global alias.br = branch
  \end{lstlisting}

\end{frame}

\subsection{Git config}
\begin{frame}[fragile]
  \subslidetitle

Lets look at our global config with \cmd{cat \char`~/.gitconfig}
\begin{lstlisting}
[user]
  name = Tux Penguin
  email = tux@penguin

[color]
  diff = auto
  status = auto
  branch = auto
  grep = auto

[alias]
  l = log --graph --pretty=format:'%C(yellow)%h%C(cyan)%d%Creset %s %C(white)- %an, %ar%Creset'
  ll = log --stat --abbrev-commit
\end{lstlisting}

\cmd{git config --list} will show the currently active config.
\end{frame}
